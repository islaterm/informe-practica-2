\section{Objetivos}
  El objetivo principal de la práctica fue crear una plataforma que permitiera a empresas
  de pequeño y mediano tamaño manejar y administrar flotas de vehículos de manera
  centralizada.
  Además la aplicación debía contar con dos tipos de usuarios: administradores y 
  conductores.

  Los administradores estaban encargados de monitorear el estado de los vehículos y las
  entregas encargadas a la empresa e informar de cualquier problema que surgiera durante
  su traslado.

  Por otro lado, la aplicación de los conductores debía enviar la ubicación del vehículo
  durante toda la ruta e informar de cualquier cambio de estado del pedido.
  Adicionalmente, se requería que tuviera integración con el \textit{GPS} del dispositivo 
  móvil y que pudiera sugerir una ruta óptima de entrega para el conductor (considerando 
  que una ruta está compuesta de una cantidad arbitraria de encargos).

  El sistema a desarrollar fue pensado como una aplicación móvil dado la versatilidad de
  esta plataforma y la comodidad que le brindaría a los usuarios.

  El sistema se planteó con una arquitectura \textit{serverless} que utilizara los
  servicios de \textit{AWS}\footnotemark.
  Junto con lo anterior, se limitaron las herramientas de desarrollo de la aplicación a
  \textit{Python} y \textit{React Native}.
  \footnotetext{\textit{Amazon Web Services}}

  El objetivo final esperado al terminar el período de práctica era la entrega de un 
  MVP\footnotemark\ que cumpliera con los requisitos mínimos para que la aplicación
  brindara valor a sus usuarios.
  \footnotetext{Producto mínimo viable} 
  \vfill
%