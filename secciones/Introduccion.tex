\section{Introducción}
  \subsection{Descripción de la empresa}
    La práctica profesional se realizó en \textit{SolarTracker}, un \textit{startup} 
    ubicado en el sector de Providencia, Santiago.
    
    La alineación de la empresa se basa en tres ejes fundamentales 
    \autocite{solar-tracker}:

    \begin{enumerate}
      \item Potenciar la generación de energías renovables para el desarrollo sostenible 
        de las necesidades industriales y sociales a nivel global.
      \item Facilitar el uso de la energía solar, optimizando la operación y mantenimiento
        mediante monitoreo inteligente y predicción de eventos de forma automática.
      \item Potenciar el desarrollo de energías fotovoltaicas en Chile y LATAM, 
        simplificando las capacidades técnicas de instalación y operación.
    \end{enumerate}
  %

  \subsection{Motivación de la empresa}
    Al ser un \textit{startup} pequeño, con poco tiempo en el rubro y relativamente poca experiencia
    y presupuesto, la empresa decide buscar practicantes para apoyar al equipo de desarrollo a 
    implementar nuevas herramientas al proyecto principal de la compañía.

    En total se buscaban dos practicantes: un estudiante de ingeniería civil eléctrica para realizar 
    un trabajo de investigación para impulsar el futuro desarrollo de la plataforma, y un estudiante 
    de ingeniería civil en computación (autor del presente informe) para adaptar el software 
    existente a nuevos estándares y tecnologías.
  %

  \subsection{Descripción general del trabajo realizado}
    Como ya se mencionó, el objetivo de la práctica era adaptar una plataforma existente para 
    utilizar nuevas tecnologías.
    En particular, migrar una base de datos a un nuevo esquema para utilizar las tecnologías de 
    \textit{BigQuery}\autocite{bigquery}, considerando todos los cambios que esta migración tenía 
    sobre la plataforma ya construida.
    
    Las tecnologías utilizadas para resolver el problema fueron el lenguaje de programación 
    \textit{Python}\autocite{python}, \textit{MongoDB}\autocite{mongo}, \textit{DAGs} de 
    \textit{Apache Airflow}\autocite{dag} y la ya mencionada \textit{BigQuery}. 
  %
%