% region SOLUCION
\section{Descripción de la solución}
  La solución al problema planteado se modeló finalmente como dos aplicaciones móviles, 
  una para el administrador del sistema y otra para los conductores.

  La aplicación del administrador le permitía manejar una pequeña flota de vehículos y 
  asignar tareas a cada uno de estos.

  Por otro lado, para los conductores el programa mostraba una visualización de las 
  tareas encomendadas por el administrador (mostrando la ruta hasta destino y los 
  detalles del pedido) que debía marcar como terminado una vez entregado. 
  Además se le daba la opción a este usuario de aceptar o rechazar una tarea.

  Para implantar la solución se utilizó el \textit{framework} \textit{serverless.com} que
  permitía una fácil configuración para distintos servicios de \textit{cloud computing}, 
  entre los cuales se encuentra \textit{AWS} mediante uno o varios archivos \textit{YAML}.

  A continuación se presentan las funcionalidades y detalles de la implementación para 
  estas aplicaciones.

  % region ADMIN
  \subsection{Aplicación del administrador}
    Para este proyecto, un administrador es un usuario que maneja a los conductores de 
    una empresa.
    Ellos se encargan de: 
    \begin{itemize}
      \item Crear y asignar tareas a conductores
      \item crear nuevos conductores y nuevos vehículos
    \end{itemize}

    Con el fin de facilitar el flujo de información de todos los vehículos, se diseñaron 
    cuatro pestañas en la parte inferior de la aplicación.
    \begin{enumerate}
      \item La primera pestaña comprendía la información de todos los conductores 
        pertenecientes al sistema.
        La implementación se hizo de tal forma que al presionar el nombre de alguno de 
        los conductores se desplegara el detalle de su estado, mostrando su nombre, 
        ubicación, tareas pendientes y el vehículo en el que se trasladaba.
      \item En la segunda pestaña se podía acceder a los detalles de los vehículos.
        La información contenida en esta sección es análoga a la del conductor.
      \item La tercera pestaña contenía la información de las tareas creadas por el 
        administrador.
        En esta sección el administrador tenía la capacidad de crear tareas nuevas y 
        asignarlas a conductores y vehículos.
      \item La última pestaña contenía reportes de uso de la aplicación y los vehículos, 
        tales como la cantidad recorrida por estos el último mes, la cantidad de tareas 
        completadas por conductor, entre otras.
    \end{enumerate}

    Para almacenar y administrar los datos se utilizaron bases de datos no relacionales, 
    específicamente el motor \textit{DynamoDB} ofrecido por \textit{AWS} dada la 
    flexibilidad que ofrecía respecto a utilizar bases de datos relacionales y su fácil 
    integración con servicios \textit{serverless}.

    Para conectar la aplicación con el sistema de \textit{cloud computing} se utilizó el 
    servicio \textit{AWS API Gateway} que permitía la comunicación entre las distintas 
    partes del programa con los servicios de \textit{Amazon}.

    La conexión con la base de datos y el manejo de la información de estas se realizó en
    \textit{Python}, modelando las \textit{requests} a la \textit{API} mediante el uso de
    \textit{AWS Lambda}. 
    Así, cada petición se estructuraba como una sola función \textit{lambda}, lo que 
    evitaba tener que manejar servidores, facilitando el modelo de computación en la nube
    de la aplicación.
    Además, esto último también facilitaba la integración de plataformas de \textit{IoT}
    al sistema.
    
    El acceso a la base de datos por parte de \textit{Python} se hacía mediante la 
    librería \textit{boto3} distribuida por \textit{Amazon}.

    Para diseñar el \textit{frontend} de la \textit{app} se utilizó \textit{React Native},
    lo que permitía crear aplicaciones nativas para plataformas móviles utilizando 
    \textit{JavaScript}.
    De esta forma, no eran necesarias implementaciones distintas para \textit{iOS} y 
    \textit{Android}, ya que el código era compilado a una aplicación nativa para cada 
    una de estas plataformas.

    El diseño de la interfaz gráfica fue destinada a especialistas en el área de 
    \textit{UX}, quienes optaron por seguir las normativas de diseño de \textit{Google}, 
    en particular las definidas en \textit{Material Design}.
    Luego, el equipo de desarrollo debía implementar las vistas diseñadas en 
    \textit{React Native}, lo que fue una de las problematicas más importantes a la hora 
    de poner en funcionamiento la plataforma.

    Si bien \textit{React Native} proveía de las funcionalidades básicas para implementar 
    aplicaciones nativas, la mayoría de las herramientas estaban construidas por la 
    comunidad por lo que carecían de soporte oficial.
    Esto último llevo a que varias de las decisiones de diseño tomadas en un comienzo 
    tuvieran que modificarse para adaptarse a las limitaciones del \textit{framework}.

    Entre las herramientas de \textit{React} utilizadas destacan: 
    \textit{react-native-paper}, que provee los elementos necesarios para desarrollar la 
    interfaz de usuario siguiendo los lineamientos de \textit{Material Design}; 
    \textit{react-native-maps}, que permitía una conexión de la aplicación con los 
    servicios de \textit{Google Maps} y \textit{Google Places}; y las librerías 
    correspondientes a los servicios de \textit{AWS} como \textit{Amplify} y 
    \textit{Cognito}.


    La conexión entre el \textit{backend} y el \textit{frontend} de la aplicación se hizo
    mediante los \textit{endpoints} generados por \textit{AWS API Gateway}.
  % endregion

  % region CONDUCTOR
  \subsection{Aplicación del conductor}
    La aplicación por parte del conductor se implementó de manera simil a la del 
    administrador, utilizando los servicios de \textit{AWS} junto con \textit{Python} 
    y \textit{React Native} para el \textit{backend} y \textit{frontend} respectivamente.

    La gran diferencia de este sistema con el presentada en la sección anterior es que el 
    conductor tiene menos herramientas disponibles.
    
    Entre las facultades que posee el conductor están las de aceptar y rechazar encargos,
    marcar algún encargo realizado como terminado y cambiar su estado a disponible o 
    inactivo.
    Además, al momento de aceptar un pedido, la aplicación decide una ruta óptima hasta 
    el destino utilizando la \textit{API} de \textit{Google Maps}.
  % endregion
  
  % region PROBLEMAS
  \subsection{Problemáticas encontradas}
    El problema más importante al momento de desarrollar la solución fue lo acotado de 
    los tiempos de entrega, debido a los plazos que se tenían y a que muchas veces los 
    días laborales se veían acortados por reuniones.

    Además, como se mencionó previamente, uno de los mayores desafíos de la 
    implementación fue el diseño de la interfaz gráfica con \textit{React Native}.
    Las principales causas de esta dificultad se debieron al desconocimiento del equipo 
    respecto a estas herramientas. 
    Lo anterior implicó dedicar gran parte del tiempo a conocer los estándares de 
    desarrollo, las herramientas de \textit{testing} y de \textit{debugging}.
  % endregion
% endregion
