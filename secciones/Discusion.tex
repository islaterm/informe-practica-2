% region DISCUSION
\section{Discusión y reflexión sobre la práctica}
  % ¿Cambios no previstos?
  Un primer aspecto a mencionar de la práctica es que en un comienzo estaba pensada para
  que los practicantes trabajaran en 2 proyectos: el \textbf{sistema de manejo de flotas} 
  del que se habla en el presente informe, y un \textbf{sistema para detectar episodios de 
  epilepsia} mediante la lectura y anlálisis de ondas cerebrales mediante el uso de 
  \textit{machine learning}.
  Finalmente solamente se trabajo en el primer proyecto mencionado debido a lo acotado de
  los tiempos y ya que le generaba un mayor valor a la empresa.

  % ¿Qué debilidades había antes de hacer la práctica? y ¿Cómo se resolvieron?
  Uno de los primeros problemas a los que se vio enfrentado el equipo fue la inexperiencia
  de todos sus miembros, dado que se trabajó con tecnologías muy nuevas (e.g. 
  \textit{React Native} fue liberado en marzo del 2015, encontrándose aún en la versión 
  0.6).
  A lo anterior se suma que en un comienzo el equipo de desarrollo estaba formado por 4 
  practicantes y un solo profesional (el supervisor de los practicantes)\footnotemark\ y 
  que no existía ninguna base sobre la que trabajar puesto que el proyecto aún no había 
  sido puesto en marcha, encomendándose esta tarea a los practicantes.
  \footnotetext{
    En los días siguientes se sumaría una nueva practicante al equipo y otros 
    profesionales, pero el equipo de desarrollo seguía formado en su mayoría por 
    estudiantes.
  }

  Adicionalmente, las primeras semanas de la práctica sirvieron como un proceso de 
  nivelación dado que no todos los miembros del equipo habían recibido la misma formación.

  Otro problema relacionado a lo acotado de los plazos de entrega fue que perjudicó el 
  trabajo al momento de plantear herramientas y modalidades alternativas para el 
  desarrollo de la aplicación, teniendo muchas veces que quedarse con herramientas 
  subóptimas dado que no había espacio para investigar sobre otras.
  Sin embargo, este problema pudo ser abordado de buena manera gracias a las metodologías
  ágiles adoptadas por el equipo, ya que permitieron una mejor comunicación y organización
  del equipo.

  Una fortaleza que vale la pena mencionar es que la formación de los practicantes, (en 
  particular los pertenecientes a la especialidad de computación) entregaba gran parte de
  los conocimientos necesarios para facilitar el aprendizaje de las tecnologías a 
  utilizar, además de haber desarrollado una mentalidad crítica para lograr plantear y 
  discutir los posibles diseños para solucionar los diversos problemas a resolver.
  Entre los cursos que fueron de utilidad para lo anterior destacan: 
  \begin{itemize}
    
    \item \textit{Metodologías de diseño y programación}, que permitió concretizar 
      soluciones con un buen diseño;
    
    \item \textit{Bases de datos}, que ayudó a organizar y comprender los datos que debía
      manejar el sistema;
    \item \textit{Ingeniería de software}, que sirvió de experiencia previa para trabajar 
      en equipo y adoptar herramientas de desarrollo con un modelo 
      \textit{cliente-servidor}, además de servir de introducción a principios ágiles; y
    \item \textit{Lenguajes de programación}, que simplificó la tarea de adaptarse a las 
      nuevas herramientas utilizadas durante el período.
  \end{itemize} 
  En esa misma línea, también se pudo notar que el haber cursado los cursos de 
  especialidad facilitó que los miembros profesionales del equipo tomaran en cuenta las 
  ideas propuestas por los practicantes.
  
  Si lugar a dudas, uno de los mayores problemas al momento de trabajar en la práctica fue
  la falta de conocimientos previos (como se ha mencionado anteriormente) acerca de las 
  tecnologías empleadas, y habría hecho más fluido el avance del proyecto si dichos 
  conocimientos hubieran existido antes de comenzar.

  A pesar de todas las dificultades encontradas, la buena comunicación del equipo y el 
  apoyo, tanto de los otros practicantes como de los trabajadores de la empresa, ayudo 
  inmensamente a que siempre lograran encontrarse soluciones a las problemáticas que 
  surgieran.

  \vfill
% endregion