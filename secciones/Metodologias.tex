\section{Metodologías}
  Para solucionar el problema planteado se siguió una metodología de investigación y 
  desarrollo.
  El proceso de investigación fue adoptado ya que gran parte de el equipo desconocía las 
  herramientqas utilizadas para implementar el producto.

  Además, el trabajo fue organizado siguiendo principios ágiles de desarrollo.
  El esquema de desarrollo planteado estaba compuesto de las siguientes etapas:
  \begin{itemize}
    \item \textit{Stand up meetings} diarios al finalizar el día de trabajo para 
      visualizar el avance del proyecto y detectar errores de manera temprana.
    \item \textit{Sprints} de dos semanas dirigidos por una \textit{Scrum Master} con 
      objetivos que eran revisados al final del \textit{sprint} en un \textit{review}.
    \item Reuniones semanales con el vicepresidente del área de corporaciones de la 
      empresa para informar de los avances de los proyectos de la célula y presentar 
      oportunidades, ideas y posibles ofertas.
  \end{itemize}

  Adicionalmente, siguiendo también principios ágiles se utilizó un kanban para organizar
  el trabajo.
  El kanban contenía \textit{historias de usuario} que representaban hitos en el 
  desarrollo del sistema.
  Luego, cada historia de usuario se dividía en tareas atómicas en las que podía trabajar 
  uno o más de los integrantes del equipo dependiendo de su dificultad.
  Todas estas tareas pasaban por una revisión de calidad antes de considerarse terminadas.

  Uno de los principales beneficios de seguir las metodologías mencionadas fue la 
  flexibilidad que otorgaba al momento de organizar y fijar los objetivos del proyecto, 
  ayudando a aclarar el trabajo que debía realizarse en cada jornada.

  Las metodologías utilizadas probaron ser efectivas, obteniéndose los resultados 
  presentados en la siguiente sección.
%