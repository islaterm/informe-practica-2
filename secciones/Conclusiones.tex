% region CONCLUSIONES
\section{Conclusiones}
  Como se mencionó en secciones anteriores, el objetivo principal de la práctica fue 
  desarrollar una primera versión del sistema de manejo de flotas y concretar un producto
  mínimo viable.
  Este producto consistía de las dos aplicaciones descritas previamente (una para el 
  administrador y otra para los conductores), con una interfaz de usuario nativa para 
  \textit{Android} y un servicio \textit{serverless} montado en la nube de \textit{AWS}
  para manejar la base de datos del programa.
  Al final del período de práctica, se logró cumplir con este objetivo, faltando solamente
  algunas funcionalidades de menor importancia (como que la aplicación pudiera tomar 
  fotografías)

  Los principales conocimientos obtenidos tienen  que ver con las tecnologías utilizadas.
  Se aprendió a trabajar con servicios \textit{serverless} y a desarrollar aplicaciones 
  móviles utilizando \textit{React Native}.
  Adicionalmente se aprendieron mecánicas de trabajo en equipo, de organización de tareas
  para trabajar en proyectos y de principios de desarrollo ágil.

  Se concluye finalmente que la práctica fue una experiencia fructífera para desarrollar
  habilidades y aprendizajes relacionados al mundo laboral de ingeniería civil en 
  computación, destacándose las competencias mencionadas en ésta y las secciones 
  anteriores del informe. 
% endregion