\documentclass[11pt]{article}

%%  Allows the user to specify an input encoding
\usepackage[utf8x]{inputenc}
%%  Provides support (total or partial) for about 200 languages
\usepackage[spanish, es-lcroman]{babel}
\usepackage[style=reading]{biblatex}
\usepackage[letterpaper]{geometry}
\usepackage{fullpage}
\usepackage{setspace}
% \usepackage[final]{pdfpages}
% \usepackage{amssymb,amsmath,amsthm,amsfonts}
% \usepackage{calc}
\usepackage{graphicx}
% \usepackage{subfigure}
% \usepackage{textcomp, gensymb}
% \usepackage{natbib}
% \usepackage{url}
% \usepackage[utf8x]{inputenc}
% \usepackage{amsmath}
\usepackage{parskip}
\usepackage{fancyhdr}
\usepackage{vmargin}
% \usepackage[colorlinks=true, linkcolor=black, urlcolor=blue, pdfborder={0 0 0}]{hyperref}
\title{Informe de Práctica Profesional}
\author{Ignacio Slater M.}
\date{\today}

%% Subtítulo del documento
\newcommand{\subtitulo}{Solar Tracker Reckoner}
%%  Código del curso
\newcommand{\codigoCurso}{CC5901}
%%  Nombre del curso
\newcommand{\nombreCurso}{Práctica Profesional II}
%%  Logo de la carrera
\newcommand{\logo}{logos/LogoDCC.png}
%%  Ciudad de escritura del informe
\newcommand{\ciudad}{Santiago}
%%  País de escritura del informe
\newcommand{\pais}{Chile}

%%  Número de teléfono
\newcommand{\telefono}{(+56) 9 9158 5187}
%%  RUT del autor
\newcommand{\rut}{19.133.399-3}
%%  Mail del autor
\newcommand{\mail}{ignacio.slater@ug.uchile.cl}

%%  Fecha de inicio de la práctica
\newcommand{\inicioPractica}{19/01/2020}
%%  Fecha de fin de la práctica
\newcommand{\finPractica}{28/02/2020}
\graphicspath{{img/}}

\makeatletter
\let\titulo\@title
\let\autor\@author
\let\fecha\@date

\def\@maketitle{
  \centering
  % Logo de la universidad	
	\includegraphics[height=3cm]{\logo} \\[1 cm]
	% Código y nombre del curso
	\Large{\textsc{\codigoCurso - \nombreCurso}}	\\[0.5 cm]
  \rule{\linewidth}{0.2 mm} \\[0.4 cm]
  % Titulo del documento
	{\huge \textbf{\titulo}}	\\[0.4 cm]
  % Subtitulo del documento
  {\huge \textit{\subtitulo}}	\\
  \rule{\linewidth}{0.2 mm} \\[1.5 cm]
  \vspace{-6ex}
  % region LOGO SOLAR TRACKER
  \begin{figure}[h]
    \centering
    \includegraphics[width=0.7\textwidth]{logos/Solar Tracker.png}    
  \end{figure}
  % endregion
	% Datos autor
	\begin{minipage}{0.4\textwidth}
    \begin{flushleft} 
      {\Large \autor}  \\[0.4 cm]
      \large{
        \textit{\rut} \\
        \mail \\
        \telefono
      }
		\end{flushleft}
  \end{minipage}
  % Datos práctica
	\begin{minipage}{0.4\textwidth}
		\begin{flushright}
      \large {
        \textbf{Duración de la práctica:} \\
        \inicioPractica\ - \finPractica
      }
		\end{flushright}	
	\end{minipage}\\[2 cm]
  \vfill
  % Datos informe
  \large{
    \fecha \\
    \ciudad, \pais  \\
  }
}
\makeatother

\pagestyle{fancy}
\fancyhf{}
\rhead{\autor}
\lhead{\titulo}
\cfoot{\thepage}

\addbibresource{Referencias.bib}

\begin{document}
	\nocite{*}
	%region	PORTADA
	\begin{titlepage}
		\maketitle
		\thispagestyle{empty}
	\end{titlepage}
	%endregion

	\onehalfspacing	%	Interlineado de 1.5
	%region RESUMEN
	\pagenumbering{Roman}
	\begin{abstract}
		En el presente informe se explica el planteamiento y la solución de un proyecto de 
		\textit{software} de manejo de flotas bajo el contexto de práctica profesional de
		ingeniería civil en computación.

		La práctica se llevó a cabo en la célula \textit{Ocean}, perteneciente a 
		\textit{Entel} y tuvo una duración aproximada de dos meses a jornada completa.

		El objetivo final de la práctica era el de entregar un producto mínimo viable del 
		sistema solicitado.
		La solución planteada fue una aplicación móvil basada en una arquitectura 
		\textit{serverless} modelada como cliente-servidor.

		La modalidad de trabajo se organizó en \textit{sprints} de una y dos semanas, y se
		rigió por principios de agilidad, donde al final de cada \textit{sprint} se 
		presentaban los avances del proyecto.

		El sistema entregado al terminar del período de práctica consistió en dos 
		aplicaciones, una para administradores del sistema, y otra para los conductores de los 
		vehículos monitoreados.
		Ambas aplicaciones fueron desarrolladas con un \textit{backend} en \textit{Python} y 
		un \textit{frontend} en \textit{ReactNative}.
	\end{abstract}
	\newpage
	%endregion
	
	%region CONTENIDOS
	\pagenumbering{arabic}
\tableofcontents  % Índice
\pagebreak

%region INTRODUCCION
\section{Introducción}
  \subsection{Descripción de la empresa}
    La práctica profesional se realizó en \textit{SolarTracker}, un \textit{startup} 
    ubicado en el sector de Providencia, Santiago.
    
    La alineación de la empresa se basa en tres ejes fundamentales 
    \autocite{solar-tracker}:

    \begin{enumerate}
      \item Potenciar la generación de energías renovables para el desarrollo sostenible 
        de las necesidades industriales y sociales a nivel global.
      \item Facilitar el uso de la energía solar, optimizando la operación y mantenimiento
        mediante monitoreo inteligente y predicción de eventos de forma automática.
      \item Potenciar el desarrollo de energías fotovoltaicas en Chile y LATAM, 
        simplificando las capacidades técnicas de instalación y operación.
    \end{enumerate}
  %

  \subsection{Motivación de la empresa}
    Al ser un \textit{startup} pequeño, con poco tiempo en el rubro y relativamente poca experiencia
    y presupuesto, la empresa decide buscar practicantes para apoyar al equipo de desarrollo a 
    implementar nuevas herramientas al proyecto principal de la compañía.

    En total se buscaban dos practicantes: un estudiante de ingeniería civil eléctrica para realizar 
    un trabajo de investigación para impulsar el futuro desarrollo de la plataforma, y un estudiante 
    de ingeniería civil en computación (autor del presente informe) para adaptar el software 
    existente a nuevos estándares y tecnologías.
  %

  \subsection{Descripción general del trabajo realizado}
    Como ya se mencionó, el objetivo de la práctica era adaptar una plataforma existente para 
    utilizar nuevas tecnologías.
    En particular, migrar una base de datos a un nuevo esquema para utilizar las tecnologías de 
    \textit{BigQuery}\autocite{bigquery}, considerando todos los cambios que esta migración tenía 
    sobre la plataforma ya construida.
    
    Las tecnologías utilizadas para resolver el problema fueron el lenguaje de programación 
    \textit{Python}\autocite{python}, \textit{MongoDB}\autocite{mongo}, \textit{DAGs} de 
    \textit{Apache Airflow}\autocite{dag} y la ya mencionada \textit{BigQuery}. 
  %
%
\pagebreak
%endregion

%region PROBLEMA
\section{Problema abordado}
  Este cambio implicaba principalmente tres problemáticas que debían ser abordadas:
      
  \begin{enumerate}
    \item Se debían migrar los datos del antiguo motor de bases de datos\autocite{firebase} al 
      nuevo.
    %
    \item Se necesitaba adaptar 
  \end{enumerate}
  Dada la gran demanda de servicios por parte de empresas a \textit{Entel}, y el aumento
  de competidores en el mercado de telefonía móvil, se vió una oportunidad de crear una
  célula de investigación y desarrollo de \textit{software} dentro de la empresa.

  El proyecto de creación de una aplicación de manejo de flotas surge por una necesidad de
  los clientes de la empresa para manejar el estado de sus vehículos.
  Al comenzar la práctica \textit{Entel} ya contaba con un sistema que otorgaba este
  servicio, pero no era propietario de la empresa, así que se requería implementar un
  sistema similar desarrollado por el nuevo equipo de desarrollo.

  El problema principal fue el de crear dicha aplicación para poder ofrecerla a los
  usuarios finales para el mes de abril de 2019.
%
\pagebreak
%endregion

% region OBJETIVOS
\section{Objetivos}
  El objetivo principal de la práctica fue crear una plataforma que permitiera a empresas
  de pequeño y mediano tamaño manejar y administrar flotas de vehículos de manera
  centralizada.
  Además la aplicación debía contar con dos tipos de usuarios: administradores y 
  conductores.

  Los administradores estaban encargados de monitorear el estado de los vehículos y las
  entregas encargadas a la empresa e informar de cualquier problema que surgiera durante
  su traslado.

  Por otro lado, la aplicación de los conductores debía enviar la ubicación del vehículo
  durante toda la ruta e informar de cualquier cambio de estado del pedido.
  Adicionalmente, se requería que tuviera integración con el \textit{GPS} del dispositivo 
  móvil y que pudiera sugerir una ruta óptima de entrega para el conductor (considerando 
  que una ruta está compuesta de una cantidad arbitraria de encargos).

  El sistema a desarrollar fue pensado como una aplicación móvil dado la versatilidad de
  esta plataforma y la comodidad que le brindaría a los usuarios.

  El sistema se planteó con una arquitectura \textit{serverless} que utilizara los
  servicios de \textit{AWS}\footnotemark.
  Junto con lo anterior, se limitaron las herramientas de desarrollo de la aplicación a
  \textit{Python} y \textit{React Native}.
  \footnotetext{\textit{Amazon Web Services}}

  El objetivo final esperado al terminar el período de práctica era la entrega de un 
  MVP\footnotemark\ que cumpliera con los requisitos mínimos para que la aplicación
  brindara valor a sus usuarios.
  \footnotetext{Producto mínimo viable} 
  \vfill
%
\pagebreak
% endregion

% region METODOLOGIAS
\section{Metodologías}
  Para solucionar el problema planteado se siguió una metodología de investigación y 
  desarrollo.
  El proceso de investigación fue adoptado ya que gran parte de el equipo desconocía las 
  herramientqas utilizadas para implementar el producto.

  Además, el trabajo fue organizado siguiendo principios ágiles de desarrollo.
  El esquema de desarrollo planteado estaba compuesto de las siguientes etapas:
  \begin{itemize}
    \item \textit{Stand up meetings} diarios al finalizar el día de trabajo para 
      visualizar el avance del proyecto y detectar errores de manera temprana.
    \item \textit{Sprints} de dos semanas dirigidos por una \textit{Scrum Master} con 
      objetivos que eran revisados al final del \textit{sprint} en un \textit{review}.
    \item Reuniones semanales con el vicepresidente del área de corporaciones de la 
      empresa para informar de los avances de los proyectos de la célula y presentar 
      oportunidades, ideas y posibles ofertas.
  \end{itemize}

  Adicionalmente, siguiendo también principios ágiles se utilizó un kanban para organizar
  el trabajo.
  El kanban contenía \textit{historias de usuario} que representaban hitos en el 
  desarrollo del sistema.
  Luego, cada historia de usuario se dividía en tareas atómicas en las que podía trabajar 
  uno o más de los integrantes del equipo dependiendo de su dificultad.
  Todas estas tareas pasaban por una revisión de calidad antes de considerarse terminadas.

  Uno de los principales beneficios de seguir las metodologías mencionadas fue la 
  flexibilidad que otorgaba al momento de organizar y fijar los objetivos del proyecto, 
  ayudando a aclarar el trabajo que debía realizarse en cada jornada.

  Las metodologías utilizadas probaron ser efectivas, obteniéndose los resultados 
  presentados en la siguiente sección.
%
\pagebreak
% endregion

% region METODOLOGIAS
% region SOLUCION
\section{Descripción de la solución}
  La solución al problema planteado se modeló finalmente como dos aplicaciones móviles, 
  una para el administrador del sistema y otra para los conductores.

  La aplicación del administrador le permitía manejar una pequeña flota de vehículos y 
  asignar tareas a cada uno de estos.

  Por otro lado, para los conductores el programa mostraba una visualización de las 
  tareas encomendadas por el administrador (mostrando la ruta hasta destino y los 
  detalles del pedido) que debía marcar como terminado una vez entregado. 
  Además se le daba la opción a este usuario de aceptar o rechazar una tarea.

  Para implantar la solución se utilizó el \textit{framework} \textit{serverless.com} que
  permitía una fácil configuración para distintos servicios de \textit{cloud computing}, 
  entre los cuales se encuentra \textit{AWS} mediante uno o varios archivos \textit{YAML}.

  A continuación se presentan las funcionalidades y detalles de la implementación para 
  estas aplicaciones.

  % region ADMIN
  \subsection{Aplicación del administrador}
    Para este proyecto, un administrador es un usuario que maneja a los conductores de 
    una empresa.
    Ellos se encargan de: 
    \begin{itemize}
      \item Crear y asignar tareas a conductores
      \item crear nuevos conductores y nuevos vehículos
    \end{itemize}

    Con el fin de facilitar el flujo de información de todos los vehículos, se diseñaron 
    cuatro pestañas en la parte inferior de la aplicación.
    \begin{enumerate}
      \item La primera pestaña comprendía la información de todos los conductores 
        pertenecientes al sistema.
        La implementación se hizo de tal forma que al presionar el nombre de alguno de 
        los conductores se desplegara el detalle de su estado, mostrando su nombre, 
        ubicación, tareas pendientes y el vehículo en el que se trasladaba.
      \item En la segunda pestaña se podía acceder a los detalles de los vehículos.
        La información contenida en esta sección es análoga a la del conductor.
      \item La tercera pestaña contenía la información de las tareas creadas por el 
        administrador.
        En esta sección el administrador tenía la capacidad de crear tareas nuevas y 
        asignarlas a conductores y vehículos.
      \item La última pestaña contenía reportes de uso de la aplicación y los vehículos, 
        tales como la cantidad recorrida por estos el último mes, la cantidad de tareas 
        completadas por conductor, entre otras.
    \end{enumerate}

    Para almacenar y administrar los datos se utilizaron bases de datos no relacionales, 
    específicamente el motor \textit{DynamoDB} ofrecido por \textit{AWS} dada la 
    flexibilidad que ofrecía respecto a utilizar bases de datos relacionales y su fácil 
    integración con servicios \textit{serverless}.

    Para conectar la aplicación con el sistema de \textit{cloud computing} se utilizó el 
    servicio \textit{AWS API Gateway} que permitía la comunicación entre las distintas 
    partes del programa con los servicios de \textit{Amazon}.

    La conexión con la base de datos y el manejo de la información de estas se realizó en
    \textit{Python}, modelando las \textit{requests} a la \textit{API} mediante el uso de
    \textit{AWS Lambda}. 
    Así, cada petición se estructuraba como una sola función \textit{lambda}, lo que 
    evitaba tener que manejar servidores, facilitando el modelo de computación en la nube
    de la aplicación.
    Además, esto último también facilitaba la integración de plataformas de \textit{IoT}
    al sistema.
    
    El acceso a la base de datos por parte de \textit{Python} se hacía mediante la 
    librería \textit{boto3} distribuida por \textit{Amazon}.

    Para diseñar el \textit{frontend} de la \textit{app} se utilizó \textit{React Native},
    lo que permitía crear aplicaciones nativas para plataformas móviles utilizando 
    \textit{JavaScript}.
    De esta forma, no eran necesarias implementaciones distintas para \textit{iOS} y 
    \textit{Android}, ya que el código era compilado a una aplicación nativa para cada 
    una de estas plataformas.

    El diseño de la interfaz gráfica fue destinada a especialistas en el área de 
    \textit{UX}, quienes optaron por seguir las normativas de diseño de \textit{Google}, 
    en particular las definidas en \textit{Material Design}.
    Luego, el equipo de desarrollo debía implementar las vistas diseñadas en 
    \textit{React Native}, lo que fue una de las problematicas más importantes a la hora 
    de poner en funcionamiento la plataforma.

    Si bien \textit{React Native} proveía de las funcionalidades básicas para implementar 
    aplicaciones nativas, la mayoría de las herramientas estaban construidas por la 
    comunidad por lo que carecían de soporte oficial.
    Esto último llevo a que varias de las decisiones de diseño tomadas en un comienzo 
    tuvieran que modificarse para adaptarse a las limitaciones del \textit{framework}.

    Entre las herramientas de \textit{React} utilizadas destacan: 
    \textit{react-native-paper}, que provee los elementos necesarios para desarrollar la 
    interfaz de usuario siguiendo los lineamientos de \textit{Material Design}; 
    \textit{react-native-maps}, que permitía una conexión de la aplicación con los 
    servicios de \textit{Google Maps} y \textit{Google Places}; y las librerías 
    correspondientes a los servicios de \textit{AWS} como \textit{Amplify} y 
    \textit{Cognito}.


    La conexión entre el \textit{backend} y el \textit{frontend} de la aplicación se hizo
    mediante los \textit{endpoints} generados por \textit{AWS API Gateway}.
  % endregion

  % region CONDUCTOR
  \subsection{Aplicación del conductor}
    La aplicación por parte del conductor se implementó de manera simil a la del 
    administrador, utilizando los servicios de \textit{AWS} junto con \textit{Python} 
    y \textit{React Native} para el \textit{backend} y \textit{frontend} respectivamente.

    La gran diferencia de este sistema con el presentada en la sección anterior es que el 
    conductor tiene menos herramientas disponibles.
    
    Entre las facultades que posee el conductor están las de aceptar y rechazar encargos,
    marcar algún encargo realizado como terminado y cambiar su estado a disponible o 
    inactivo.
    Además, al momento de aceptar un pedido, la aplicación decide una ruta óptima hasta 
    el destino utilizando la \textit{API} de \textit{Google Maps}.
  % endregion
  
  % region PROBLEMAS
  \subsection{Problemáticas encontradas}
    El problema más importante al momento de desarrollar la solución fue lo acotado de 
    los tiempos de entrega, debido a los plazos que se tenían y a que muchas veces los 
    días laborales se veían acortados por reuniones.

    Además, como se mencionó previamente, uno de los mayores desafíos de la 
    implementación fue el diseño de la interfaz gráfica con \textit{React Native}.
    Las principales causas de esta dificultad se debieron al desconocimiento del equipo 
    respecto a estas herramientas. 
    Lo anterior implicó dedicar gran parte del tiempo a conocer los estándares de 
    desarrollo, las herramientas de \textit{testing} y de \textit{debugging}.
  % endregion
% endregion

\pagebreak
% endregion

% region DISCUSION
% region DISCUSION
\section{Discusión y reflexión sobre la práctica}
  % ¿Cambios no previstos?
  Un primer aspecto a mencionar de la práctica es que en un comienzo estaba pensada para
  que los practicantes trabajaran en 2 proyectos: el \textbf{sistema de manejo de flotas} 
  del que se habla en el presente informe, y un \textbf{sistema para detectar episodios de 
  epilepsia} mediante la lectura y anlálisis de ondas cerebrales mediante el uso de 
  \textit{machine learning}.
  Finalmente solamente se trabajo en el primer proyecto mencionado debido a lo acotado de
  los tiempos y ya que le generaba un mayor valor a la empresa.

  % ¿Qué debilidades había antes de hacer la práctica? y ¿Cómo se resolvieron?
  Uno de los primeros problemas a los que se vio enfrentado el equipo fue la inexperiencia
  de todos sus miembros, dado que se trabajó con tecnologías muy nuevas (e.g. 
  \textit{React Native} fue liberado en marzo del 2015, encontrándose aún en la versión 
  0.6).
  A lo anterior se suma que en un comienzo el equipo de desarrollo estaba formado por 4 
  practicantes y un solo profesional (el supervisor de los practicantes)\footnotemark\ y 
  que no existía ninguna base sobre la que trabajar puesto que el proyecto aún no había 
  sido puesto en marcha, encomendándose esta tarea a los practicantes.
  \footnotetext{
    En los días siguientes se sumaría una nueva practicante al equipo y otros 
    profesionales, pero el equipo de desarrollo seguía formado en su mayoría por 
    estudiantes.
  }

  Adicionalmente, las primeras semanas de la práctica sirvieron como un proceso de 
  nivelación dado que no todos los miembros del equipo habían recibido la misma formación.

  Otro problema relacionado a lo acotado de los plazos de entrega fue que perjudicó el 
  trabajo al momento de plantear herramientas y modalidades alternativas para el 
  desarrollo de la aplicación, teniendo muchas veces que quedarse con herramientas 
  subóptimas dado que no había espacio para investigar sobre otras.
  Sin embargo, este problema pudo ser abordado de buena manera gracias a las metodologías
  ágiles adoptadas por el equipo, ya que permitieron una mejor comunicación y organización
  del equipo.

  Una fortaleza que vale la pena mencionar es que la formación de los practicantes, (en 
  particular los pertenecientes a la especialidad de computación) entregaba gran parte de
  los conocimientos necesarios para facilitar el aprendizaje de las tecnologías a 
  utilizar, además de haber desarrollado una mentalidad crítica para lograr plantear y 
  discutir los posibles diseños para solucionar los diversos problemas a resolver.
  Entre los cursos que fueron de utilidad para lo anterior destacan: 
  \begin{itemize}
    
    \item \textit{Metodologías de diseño y programación}, que permitió concretizar 
      soluciones con un buen diseño;
    
    \item \textit{Bases de datos}, que ayudó a organizar y comprender los datos que debía
      manejar el sistema;
    \item \textit{Ingeniería de software}, que sirvió de experiencia previa para trabajar 
      en equipo y adoptar herramientas de desarrollo con un modelo 
      \textit{cliente-servidor}, además de servir de introducción a principios ágiles; y
    \item \textit{Lenguajes de programación}, que simplificó la tarea de adaptarse a las 
      nuevas herramientas utilizadas durante el período.
  \end{itemize} 
  En esa misma línea, también se pudo notar que el haber cursado los cursos de 
  especialidad facilitó que los miembros profesionales del equipo tomaran en cuenta las 
  ideas propuestas por los practicantes.
  
  Si lugar a dudas, uno de los mayores problemas al momento de trabajar en la práctica fue
  la falta de conocimientos previos (como se ha mencionado anteriormente) acerca de las 
  tecnologías empleadas, y habría hecho más fluido el avance del proyecto si dichos 
  conocimientos hubieran existido antes de comenzar.

  A pesar de todas las dificultades encontradas, la buena comunicación del equipo y el 
  apoyo, tanto de los otros practicantes como de los trabajadores de la empresa, ayudo 
  inmensamente a que siempre lograran encontrarse soluciones a las problemáticas que 
  surgieran.

  \vfill
% endregion
\pagebreak
% endregion

% region CONCLUSIONES
% region CONCLUSIONES
\section{Conclusiones}
  Como se mencionó en secciones anteriores, el objetivo principal de la práctica fue 
  desarrollar una primera versión del sistema de manejo de flotas y concretar un producto
  mínimo viable.
  Este producto consistía de las dos aplicaciones descritas previamente (una para el 
  administrador y otra para los conductores), con una interfaz de usuario nativa para 
  \textit{Android} y un servicio \textit{serverless} montado en la nube de \textit{AWS}
  para manejar la base de datos del programa.
  Al final del período de práctica, se logró cumplir con este objetivo, faltando solamente
  algunas funcionalidades de menor importancia (como que la aplicación pudiera tomar 
  fotografías)

  Los principales conocimientos obtenidos tienen  que ver con las tecnologías utilizadas.
  Se aprendió a trabajar con servicios \textit{serverless} y a desarrollar aplicaciones 
  móviles utilizando \textit{React Native}.
  Adicionalmente se aprendieron mecánicas de trabajo en equipo, de organización de tareas
  para trabajar en proyectos y de principios de desarrollo ágil.

  Se concluye finalmente que la práctica fue una experiencia fructífera para desarrollar
  habilidades y aprendizajes relacionados al mundo laboral de ingeniería civil en 
  computación, destacándose las competencias mencionadas en ésta y las secciones 
  anteriores del informe. 
% endregion
\pagebreak
% endregion

\printbibliography[heading=bibintoc, title=Bibliografías]
	%endregion
\end{document} 	